% Chapter Template

\chapter{Ensayos y resultados} % Main chapter title

\label{Chapter4} % Change X to a consecutive number; for referencing this chapter elsewhere, use \ref{ChapterX}


%----------------------------------------------------------------------------------------
%	SECTION 1
%----------------------------------------------------------------------------------------
En este capítulo se detallan los resultados esperados y obtenidos sobre cada una de las pruebas realizadas para validar la integración del sistema y poder comprobar que el alcance funcional logrado es acorde a los esperado.

%\citep{ARTICLE:1}, \citep{BOOK:1}, \citep{BOOK:2}, \citep{WEBSITE:1}.

\section{Banco de pruebas}

Todos los ensayos que se describen en este capítulo fueron efectuados utilizando el diseño físico de red que se muestra en la figura \ref{fig:banco}. Las pruebas funcionales desde dentro de la red local se realizaron con una laptop MacBook Pro y un equipo de escritorio con Windows 10. Para las pruebas de funcionalidad remota se realizaron utilizando un dispositivo móvil Samsung A50 con acceso a la red celular mediante el uso de paquetes de datos a Internet.

\begin{figure}[htbp]
	\centering
	\includegraphics[width=0.82\textwidth]{./Figures/banco2.png}
	\caption{Esquema del banco de pruebas utilizado.}

	\label{fig:banco}
\end{figure}


\section{Pruebas de elección de canal y ancho de banda}
Para las pruebas y análisis de las señales inalámbricas se utilizó el software WiFi Explorer Lite. Es una herramienta de descubrimiento de redes inalámbricas que ayuda a identificar conflictos de canales y problemas de configuración y  pueden afectar la conectividad o el rendimiento de la red Wi-Fi de un hogar u oficina \citep{WEBSITE:24}. 

Los fundamentos y consideraciones para la elección del canal y ancho de banda de la señal Wi-Fi IoT que se utilizó, se describió en el capitulo 3. La elección dependió de las señales circundantes vecinas a la red domestica donde se instaló el sistema prototipo IoT. La figura \ref{fig:test01} muestra resultado del primer escaneo de las señales y el uso de los canales respectivos así como solapamiento existente entre ellos. 

De la figura \ref{fig:test01} se describe lo siguiente:

\keyword{Señal WLAN IoT} 
\begin{itemize}
\item SSID: MATRIX-ICF
\item Canal: 10 (configuración automática)
\item Ancho de canal: 40 MHz (configuración automática)
\item Potencia señal: 93\%
\item Seguridad:  WPA2 (PSK)
\item Tasa máxima de transferencia: 300 Mbps
\end{itemize}


\keyword{Señal WLAN doméstica}
\begin{itemize} 
\item SSID: CLARO-B612-D514
\item Canal: 11 (configuración automática)
\item Ancho de canal: 20 MHz (configuración automática)
\item Potencia señal: 64\%
\item Seguridad: WPA2 (PSK)
\item Tasa Máxima de transferencia: 144.4 Mbps
\end{itemize}

El procedimiento de mejora de la configuración de la señal IoT consistió en modificar la configuración por defecto del router/AP al cambiar el canal y reducir el ancho de banda, verificando en cada cambio el comportamiento de las señales en el ambiente y la reducción de solapamiento de los mismos.

La figura \ref{fig:test02} muestra el ancho de banda que ocupa el canal de comunicación de la señal del router sin configurar. Esta configuración por defecto demuestra no ser la más adecuada para el ambiente IoT, debido a que produce interferencias a las señales circundantes.

La figura \ref{fig:test03} muestra las características de la señal inalámbrica doméstica del lugar donde se implanto el sistema IoT prototipo.


%%%%%%%%%%%%%%%%%%%%%%%%%%%%%%%%%%%%%%%%%%%%%%%%%%%
\begin{landscape} % esto es para rotar la pagina e imagen
\begin{figure}[htpb]
\centering 
\includegraphics[width=1.5\textwidth]{./Figures/wifi/01.png}
\caption{Estado inicial de las señales Wi-Fi local y circundantes en el ambiente donde se implanto el sistema IoT.}
\label{fig:test01}
\end{figure}
\end{landscape} %

%%%%%%%%%%%%%%%%%%%%%%%%%%%%%%%%%%%%%%%%%%%%%%%%%%%

\begin{landscape} % esto es para rotar la pagina e imagen
\begin{figure}[htpb]
\centering 
\includegraphics[width=1.5\textwidth]{./Figures/wifi/02.png}
\caption{Ancho de banda de la señal del router para la red IoT con la configuración por defecto del dispositivo.}
\label{fig:test02}
\end{figure}
\end{landscape} %


%%%%%%%%%%%%%%%%%%%%%%%%%%%%%%%%%%%%%%%%%%%%%%%%%%%
\begin{landscape} % esto es para rotar la pagina e imagen
\begin{figure}[htpb]
\centering 
\includegraphics[width=1.5\textwidth]{./Figures/wifi/03.png}
\caption{Ancho de banda  y canal de la señal Wi-Fi doméstica.}
\label{fig:test03}
\end{figure}
\end{landscape} %

%%%%%%%%%%%%%%%%%%%%%%%%%%%%%%%%%%%%%%%%%%%%%%%%%%%%%%%%%%%%%%
El objetivo de usar un software de exploración Wi-Fi es detectar las zonas y canales con mayor interferencia y evitar que el canal elegido para nuestro trabajo tenga la mínima o ninguna intersección con la zona critica. El software detectó la zona con mayor solapamiento y lo marcó en color rojo, como se muestra en la figura \ref{fig:test04}, asociado al SSID y canal que lo causa.

El análisis de las imágenes de señales que genera el software de exploración, nos permitió conocer cuales podrían ser los canales mas ideales. Cambiar el numero de canal así como el ancho de banda del mismo, tiene como objetivo reducir o eliminar problemas de solapamiento en el canal inalámbrico usado para la red IoT. El resultado del cambio de canal para la señal destinada a la comunicación IoT, se muestra en la figura \ref{fig:test05}. 

Si se compara la figura \ref{fig:test02} (antes) con la figura \ref{fig:test05} (después), se puede observar la diferencia del canal configurado al mostrar la reducción de interferencias con las señales circundantes.

Al cambiar el canal de la señal IoT, el software cambia el color de la zona crítica (de rojo a naranja) en señal que el solapamiento se redujo, demostrando que existe una mejora en la señal de comunicación a utilizar. La figura \ref{fig:test06} muestra los resultados de mejora obtenidos.

Las pruebas y cambios de canal para el router/AP se debe realizar durante la instalación y puesta en marcha al sistema IoT y podrá ser actualizado de acuerdo al cronograma establecido para su mantenimiento. 


%%%%%%%%%%%%%%%%%%%%%%%%%%%%%%%%%%%%%%%%%%%%%%%%%%%

\begin{landscape} % esto es para rotar la pagina e imagen
\begin{figure}[htpb]
\centering 
\includegraphics[width=1.5\textwidth]{./Figures/wifi/04.png}
\caption{Zona crítica con mayor interferencia entre los canales de las redes inalámbricas.}
\label{fig:test04}
\end{figure}
\end{landscape} %


%%%%%%%%%%%%%%%%%%%%%%%%%%%%%%%%%%%%%%%%%%%%%%%%%%%

\begin{landscape} % esto es para rotar la pagina e imagen
\begin{figure}[htpb]
\centering 
\includegraphics[width=1.5\textwidth]{./Figures/wifi/05.png}
\caption{Ancho de banda y nuevo canal de funcionamiento para las señales IoT y doméstica.}
\label{fig:test05}
\end{figure}
\end{landscape} %


%%%%%%%%%%%%%%%%%%%%%%%%%%%%%%%%%%%%%%%%%%%%%%%%%%%

\begin{landscape} % esto es para rotar la pagina e imagen
\begin{figure}[htpb]
\centering 
\includegraphics[width=1.5\textwidth]{./Figures/wifi/06.png}
\caption{Zona con reducción de solapamiento después de la configuración manual del router.}
\label{fig:test06}
\end{figure}
\end{landscape} %


\section{Pruebas del módulo de temperatura}

El módulo permite leer las variables físicas de temperatura y humedad del ambiente donde se instaló. La temperatura se muestra en su pantalla Oled y con mas detalle desde el software de monitoreo y control.

La figura \ref{fig:test-temp} muestra la instalación para las pruebas y el valor obtenido en la pantalla oled. La figura \ref{fig:temp-lectura} muestra los valores obtenidos en el software de monitoreo y sus detalles se muestra en la figura \ref{fig:temp-detalle}.

\begin{figure}[htpb]
\centering 
\includegraphics[width=0.9\textwidth]{./Figures/test/temp/test-temp.png}
\caption{Funcionamiento del módulo de temperatura.}
\label{fig:test-temp}
\end{figure}

La figura \ref{fig:test-panel} muestra las características del panel de visualización del módulo en el software.

\begin{figure}[htpb]
\centering 
\includegraphics[width=0.65\textwidth]{./Figures/test/temp/panel.png}
\caption{Características del panel de visualización.}
\label{fig:test-panel}
\end{figure}

%%%%%%%%%%%%%%%%%%%%%%%%%%%%%%%%%%%%%%%%%%%%%%%%%%%

\begin{landscape} % esto es para rotar la pagina e imagen
\begin{figure}[htpb]
\centering 
\includegraphics[width=1.7\textwidth]{./Figures/test/temp/lectura.png}
\caption{Monitoreo de los módulos la temperatura en el software.}
\label{fig:temp-lectura}
\end{figure}
\end{landscape} %
%%%%%%%%%%%%%%%%%%%%%%%%%%%%%%%%%%%%%%%%%%%%%%%%%%%


\begin{landscape} % esto es para rotar la pagina e imagen
\begin{figure}[htpb]
\centering 
\includegraphics[width=1.7\textwidth]{./Figures/test/temp/detalle.png}
\caption{Detalle del módulo de temperatura conectado y activo.}
\label{fig:temp-detalle}
\end{figure}
\end{landscape} %
%%%%%%%%%%%%%%%%%%%%%%%%%%%%%%%%%%%%%%%%%%%%%%%%%%%

\section{Pruebas del módulo actuador}
Para las pruebas se usó un esquema de conexión como se muestra en la figura \ref{fig:test-esquema}, usando un ventilador como electrodoméstico indicador de consumo. 
\vspace{0.5cm}
\begin{figure}[htpb]
\centering 
\includegraphics[width=0.87\textwidth]{./Figures/test/consumo/esquema.png}
\caption{Esquema de conexión para pruebas del módulo.}
\label{fig:test-esquema}
\end{figure}

Al implementar el módulo y realizar las pruebas respectivas demostró que el sensor AC-ZMPT101B es muy sensible en la captura del valor de la tensión. El valor obtenido se contrastó con un multímetro digital dando una diferencia aproximada de +2 V / -2 V. La figura \ref{fig:test-tension} muestra la comprobación de la tensión.

\begin{figure}[htpb]
\centering 
\includegraphics[width=1.0\textwidth]{./Figures/test/consumo/tension2.png}
\caption{Comparación de la medida de la tensión entre el módulo y el multímetro.}
\label{fig:test-tension}
\end{figure}

El modulo permite leer las variables físicas de tensión e intensidad así como el estado del relé actuador. Las lecturas se muestran en su pantalla gráfica como se muestra en la figura \ref{fig:test-activa1} y en la figura \ref{fig:test-activa2}.
\vspace{0.5cm}
\begin{figure}[htpb]
\centering 
\includegraphics[width=1.0\textwidth]{./Figures/test/consumo/paso.png}
\caption{Módulo con paso de la corriente eléctrica (led rojo indica riesgo eléctrico en la toma de corriente ).}
\label{fig:test-activa1}
\end{figure}

\vspace{0.5cm}
\begin{figure}[htpb]
\centering 
\includegraphics[width=1.0\textwidth]{./Figures/test/consumo/bloqueo.png}
\caption{Módulo con bloqueo de la corriente eléctrica (led azul indica sin riesgo eléctrico en la toma de corriente).}
\label{fig:test-activa2}
\end{figure}

\vspace{0.5cm}
\section{Pruebas de consumo de energía eléctrica}

El consumo que se ha medido en estas pruebas se realizó a un ventilador de casa, cuyas características principales son:

\begin{itemize}
\item Marca: ELECTROLUX
\item Tipo:	circuladores de aire con temporizador
\item Modelo: BFV10
\item Número de velocidades: 3
\item Potencia: 30 W
\end{itemize}

La figura \ref{fig:ventilador}  muestra las especificaciones técnicas adherido en su parte posterior. Resaltar que el ventilador usado tiene más de 7 años de uso y en la actualidad este modelo de ventilador aún se comercializa, pero con un incremento en el numero de velocidades y en la potencia. 

\begin{figure}[htpb]
\centering 
\includegraphics[width=0.7\textwidth]{./Figures/test/consumo/ventilador.png}
\caption{Etiqueta con información técnica del ventilador.}
\label{fig:ventilador}
\end{figure}

Según la guía del organismo de supervisor de la inversión en energía (OSINERG - Perú) \citep{BOOK:3}, para calcular el consumo eléctrico el valor de la potencia  debe ser convertida a Kilowatts (kW), para ello se divide la potencia entre 1000. Para el ventilador usado en la pruebas el valor ideal sería la ecuación \ref{eq:potenciatest2}.

\begin{equation}
	\label{eq:potenciatest2}
	PE = \left( 0.03 \right) kW
\end{equation}

Para comparar el valor obtenido en la formula \ref{eq:potenciatest2} el módulo de consumo nos permite obtener un valor real de la potencia del ventilador mediante la formular \ref{eq:potenciaform}, las variables de tensión e intensidad de corriente son multiplicados y se obtiene un valor de potencia en un instante de tiempo. La figura \ref{fig:registroPotencia} muestra las lecturas almacenadas en la base de datos del sistema IoT y en la tabla \ref{tab:tablapotencias} se muestra la comparación del valor ideal con el valor real obtenidos desde el módulo de consumo.

\begin{figure}[htpb]
\centering 
\includegraphics[width=1.0\textwidth]{./Figures/test/consumo/lecturas.png}
\caption{Lecturas de potencia almacenadas en la base de datos del sistema IoT.}
\label{fig:registroPotencia}
\end{figure}

%\vspace{1.0cm}

\begin{table}[h]
	\centering
	\caption[Comparativa de registros de potencias obtenidas]{Comparativa de registros de potencias obtenidas}
	\begin{tabular}{l c c c}    
		\toprule
		\textbf{P. ideal (W)} 	 & \textbf{P. ideal (kW)}  & \textbf{P. real - módulo (W)} &\textbf{P. real módulo (kW)} \\
		\midrule
		30 & 0.03 & 31.35 & 0.03135\\		
		30& 0.03 & 31.55  &0.03155 \\
		30& 0.03 & 31.47 & 0.03147\\		
		30& 0.03 & 31.36 & 0.03136\\		
		30& 0.03 & 31.33 & 0.03133\\
		\bottomrule
		\hline
	\end{tabular}
	\label{tab:tablapotencias}
\end{table}

El valor de las variables físicas de tensión e intensidad que son necesarias para calcular la  potencia del ventilador usando los sensores, se muestran en el software de monitoreo y control dentro de un conjunto de paneles según la cantidad de módulos registrados en el sistema IoT.

Para facilitar la comprensión de la interfaz gráfica del software, la figura \ref{fig:test-panel5} muestra las características del panel de visualización del módulo en el software cuando el actuador permite el paso de la corriente eléctrica.

\begin{figure}[htpb]
\centering 
\includegraphics[width=1.0\textwidth]{./Figures/test/consumo/panel5.png}
\caption{Panel de visualización del módulo con paso de corriente eléctrica.}
\label{fig:test-panel4}
\end{figure}

La figura \ref{fig:test-panel4} muestra las características del panel de visualización del módulo en el software cuando el actuador bloquea el paso de la corriente eléctrica .

\begin{figure}[htpb]
\centering 
\includegraphics[width=1.0\textwidth]{./Figures/test/consumo/panel4.png}
\caption{Panel de visualización del módulo con bloqueo de la corriente eléctrica.}
\label{fig:test-panel5}
\end{figure}

Para mostrar el monto a pagar por el usuario se consideró consumos facturados y consumos no facturados. Los consumos facturados son aquellos registros resumen de un conjunto de registros temporales que se dieron dentro de un intervalo de tiempo. Por ejemplo dentro del intervalo '28-03-2022 16:00:00' y '28-03-2022 17:00:00' se capturan aproximadamente 1800 lecturas de potencia, para considerar la facturación se obtiene la media aritmética del valor de potencia del conjunto y se almacena como único registro en la tabla de facturación con fecha y hora '28-03-2022 17:00:00', posteriormente se procede a eliminar los 1800 registros temporales. Los consumos no facturados son aquellos registros temporales.

Para el calcular el monto a pagar se consideró el costo de S/0.5 nuevos soles por cada 1 kWh.

Para el ventilador de pruebas se realizó un muestreo de 5 horas de funcionamiento continuo y se obtuvieron los resultados que se muestran en la figura \ref{fig:dashboard-consumo} El dashboard del software de monitoreo y control ofrece una vista resumida del subtotal y total a pagar así como un gráfico interactivo de potencia vs costo de consumo por hora.


 %%%%%%%%%%%%%%%%%%%%%%%%%%%%%%%%%%%%%%%%%%%%%%%%%%%
\begin{landscape} % esto es para rotar la pagina e imagen
\begin{figure}[htpb]
\centering 
\includegraphics[width=1.7\textwidth]{./Figures/test/consumo/consumo.png}
\caption{Dashboard de facturación del software de monitoreo y control.}
\label{fig:dashboard-consumo}
\end{figure}
\end{landscape} %
%%%%%%%%%%%%%%%%%%%%%%%%%%%%%%%%%%%%%%%%%%%%%%%%%%%

















--------------------

Las variables de tensión, intensidad y potencia eléctrica del electrodoméstico conectado a el, se pueden visualizar en el software de monitoreo y control así como los detalles de su funcionamiento en tiempo real. La figura \ref{fig:test-activa1} y \ref{fig:test-activa2} ilustran lo mencionado.
\section{Pruebas del funcionamiento del módulo replicador}

estado de internet

replicas envía a la nube

recibe de la nube

verifica sin replicar


\section{Pruebas del funcionamiento del sistema sin Internet}


\section{Pruebas funcionales del sistema desde acceso remoto}
