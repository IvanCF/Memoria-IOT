% Chapter Template

\chapter{Conclusiones} % Main chapter title

\label{Chapter5} % Change X to a consecutive number; for referencing this chapter elsewhere, use \ref{ChapterX}

En este capítulo se detallan las conclusiones relacionadas al alcance de los objetivos que se plantearon al inicio del trabajo. Además se analizan las características de software y hardware del prototipo desarrollado, el cumplimiento de la planificación y los próximos pasos a seguir para mejorar y hacerlo un producto
comercial.

%----------------------------------------------------------------------------------------

%----------------------------------------------------------------------------------------
%	SECTION 1
%----------------------------------------------------------------------------------------

\section{Conclusiones generales }

Este trabajo logro desarrollar de forma exitosa un sistema IoT para monitoreo y control de viviendas y edificios orientados al consumo de energía eléctrica. Se verificó el cumplimiento de los requerimientos más importantes, quedando aún por validar algunos menos relevantes y que pueden ser implementados en trabajos futuros con un cronograma de mayor margen de tiempo.

Los módulos que requirieron de hardware y firmware para cumplir su función dentro del sistema IoT fueron desarrollados considerando los requerimientos funcionales y el tiempo del cronograma, como principal objetivo a cumplir, lo que puede ser mejorado en forma y diseño en futuros desarrollos.

El software principal de monitoreo desarrollado a medida para este trabajo se considera la primera versión demo del producto. Se le realizaron pruebas principales de usabilidad, funcionalidad, acceso de red y seguridad web mínimos y necesarios para poder comprobar su funcionamiento en un ambiente real. 

El desarrollo de todos los componentes de software desarrollados para el sistema IoT fueron creados y testeados dentro de un entorno operativo Windows 10, GNU/Linux Elementary OS y RasberryPi OS. 

Para mayor información y seguimiento de futuros desarrollos funcionales, se puede acceder a la web oficial: \url{https://www.cenergy.icfnet.org/}

Por lo tanto, se concluye que los objetivos planteados al inicio del trabajo han sido alcanzados satisfactoriamente y se han obtenido y reforzado conocimientos valiosos para la formación profesional del autor.


%----------------------------------------------------------------------------------------
%	SECTION 2
%----------------------------------------------------------------------------------------
\section{Próximos pasos}

Para dar continuidad al esfuerzo realizado hasta el momento y poder realizar un producto comercialmente atractivo surge la necesidad de rediseñar cada módulo físico y unificar los componentes electrónicos internos en una PCB (\emph{printed circuit board}) más pequeña, considerando estándares de fabricación de placas electrónicas para uso comercial.

Implementar nuevas funciones de ciberseguridad web para el software de monitoreo local y remoto considerando los ataques cibernéticos más comunes en dicho entorno.

Desarrollar la autenticación por token vía SMS para la validación de acceso al software principal de monitoreo y control, para garantizar una capa de seguridad web adicional al sistema actual.

Desarrollar una aplicación móvil para entornos Android y IOS para facilitar el acceso y agregado de módulos al sistema, para garantizar un servicio más amigable al usuario.

Implementar mecanismos de cifrado de unidades internas del microSD del módulo principal donde se almacena el software de monitoreo, procesos internos de red y base de datos local, para evitar la fácil manipulación de información confidencial del sistema.

